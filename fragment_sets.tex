\documentclass[12pt,a4paper,english]{article}
\usepackage{ucs}
\usepackage[utf8x]{inputenc}%utf-8 utf8x
%\usepackage{lmodern}
\usepackage{fontenc}[T1]
\usepackage{babel}%[T1] \usepackage{amssymb,amsmath,wick}
\usepackage{epsfig}
\usepackage{wrapfig}
\usepackage{color,subfigure}  
\usepackage{amsmath}
%\usepackage{beamerthemesplit}
\usepackage{graphicx}
\usepackage{listings}
%\usepackage[pdftex,colorlinks=true,bookmarks=true,linkcolor=blue]{hyperref}
\let\Tiny=\tiny

\newcommand{\be}{ \begin{equation}}
\newcommand{\ee}{ \end{equation}}
\newcommand{\mc}{ \mathcal }
\newcommand{\mbf}{ \mathbf }
\newcommand{\bra}[1]{\langle #1|}
\newcommand{\ket}[1]{|#1\rangle}
\newcommand{\braket}[2]{\langle #1|#2\rangle}
\newcommand{\braopket}[3]{\langle #1|#2|#3\rangle}
\newcommand{\beq}{\begin{equation*}}
\newcommand{\eeq}{\end{equation*}}
\newcommand{\ds}{\displaystyle{\not}}
\newcommand{\matr}[1]{{\bf \cal{#1}}}
\newcommand{\OP}[1]{{\bf\widehat{#1}}}

\newcommand{\clearemptydoublepage}{\newpage{\pagestyle{empty}\cleardoublepage}}
\setlength{\parindent}{0in} %Ingen indent

\title{Notes about the fragment sets in DEC CC.} 
\date{} 
\begin{document} In
the DEC model the HF orbitals are are assigned to atomic sites according to the
Mulliken the largest Mulliken charge. Each atomic site get assigned a set of
occupied and virtual orbitals.  An orbital corresponding to atomic site $P$
belong to the set $\underline P$ if it is occupied and to $\overline P$ if it is a
virtual orbital. \\

The correlation energy obtained from the CC equations is thus given by

\begin{equation}
  E_{corr}=\sum_{P} E_P +sum_{P>Q}\Delta E_{PQ}
  \label{eq:cccorrPQ}
\end{equation}
where $E_P$ is defined by

\begin{equation}
  E_P=\sum_{ij\in \underline P} (t^{ab}_{ij}+t^a_it^b_j)(2g_{iajb}-g_{ibja})
  \label{eq:EP}
\end{equation}
and $\Delta E_{PQ}$ is defined by
\begin{equation}
  E_{PQ}-E_P-E_Q
  \label{eq:DEPQ}
\end{equation}
and 
\begin{equation}
  E_{PQ}=\sum_{\substack{ij \in \underline P \cup \underline Q \\ab}} (t^{ab}_{ij}+t^a_it^b_j)(2g_{iajb}-g_{ibja})
  \label{eq:EPQ}
\end{equation}

The set $[\underline{P}]$ is the set of all occupied orbitals local to site $P$, while the set $[\overline{P}$ is the set of all virtual orbitals local to site $P$.
  The local HF orbitals leads to an electron correlation energy where the contributions from the term $g_{iajb}$ is nonnegligible only when 
  \begin{equation}
	i\in \underline{P},\, a\in [\overline{P}]\, \mbox{and}\, j\in \underline{Q},\, b\in [\overline{Q}].
	\label{eq:critnonneg}
  \end{equation}

This relation leads to restrictions on the sums over the virtual particles $a$ and $b$,
\begin{equation}
  \begin{split}
  & E_P=\sum_{\substack{ij\in \underline{P}\\ab\in [\overline{P}]}}(t^{ab}_{ij}+t^a_it^b_j)(2g_{iajb}-g_{ibja})\\
  & E_{PQ}=\sum_{\substack{ij\in \underline{P}\cup \underline{Q}\\ab\in [\overline{P}]\cup [\overline{Q}]}}(t^{ab}_{ij}+t^a_it^b_j)(2g_{iajb}-g_{ibja}).
\end{split}
  \label{eq:restEP}
\end{equation}

\section{Amplitude Equations}

For detailed analysis of the MP2 amplitude equations and the solution see Ref. \cite{Kristensen2011}.
We observe that the MP2 amplitude equation is expressed as
\begin{equation}
  \Omega_{aibj}=\braopket{\substack{\overline{ab}\\ ij}}{\hat H}{HF}+\braopket{\substack{\overline{ab}\\ ij}}{[\hat F,\hat T_2]}{HF}=0,
  \label{eq:MP2ampleq}
\end{equation}
where $\hat H$ is the Hamiltionian and $T_2$ is the cluster doubles operator
\begin{equation}
  \hat T_2=\frac{1}{2}\sum_{aibj}t^{ab}_{ij}E_{ai}E_{bj}.
  \label{eq:t2ampl}
\end{equation}
The operator $\hat F$ is the fock operator and its matrix elements are given as
\begin{equation}
  F_{pq}=h_{pq}+\sum_{i}(2g_{pqii}-g_{piiq}).
  \label{eq:fockopele}
\end{equation}
Because of Brillouin's theorem the fock matrix will be on a block diagonal form where the occupied-virtual block yields zero
\begin{equation}
  F_{OV}=0.
  \label{eq:fockoccvirt}
\end{equation}
The process to solve Eq. \eqref{eq:MP2ampleq} is solved by standard iterative algorithms, such as the conjugate gradient or conjugate residual methods, see 
Ref. \cite{book:helgco} after using one of the above mentioned algorithms a new direction for the $(n+1)$'th iteration is determined from the residual 

\begin{equation}
  R^n_{aibj}=0-\Omega_{aibj}=-\Omega_{aibj},
  \label{eq:residuedef}
\end{equation}
and the amplitude in the $(n+1)$'th iteration is updated according to 
\begin{equation}
  t^{n+1}=t^n+R^n.
  \label{eq:n+1iteration}
\end{equation}
When $R^n$ is lower than a given threshold the iteration process is completed.

\section{The Fragments in the Amplitude Equation}

The MP2 amplitude equation is finally expressed by
\begin{equation}
  \sum_c t^{cb}_{ij}F_{ac}+\sum_ct^{ac}F_{bc}-\sum_kt^{ab}_{kj}F_{ki}-\sum_k t^{ab}_{ik}F_{kj}+g_{aibj}=0.
  \label{eq:ampleqexp}
\end{equation}
%with the solution
%\begin{equation}
%   \sum_c t^{cb}_{ij}F_{ac}+\sum_ct^{ac}F_{bc}-\sum_kt^{ab}_{kj}F_{ki}-\sum_k t^{ab}_{ik}F_{kj}=-g_{aibj}.
%  \label{eq:sovamplmp2}
%\end{equation}
Because of the deacrising properties of the fock matrix elements with increasing distances between the molecular orbitals, the off-diagonal Fock matrix elements are non-negligible only if
\begin{equation}
  \begin{split}
	&F^{oo}_{kl}; k\in \underline{P},l\in [\underline{P}]~\mbox{or}~k\in[\underline{P}],l\in \underline{P}\\
	&F^{vv}_{ab};a\in\overline{P},b\in[\overline{P}]~\mbox{or}~a\in]\overline{P}],b\in\overline{P}.
  \end{split}
  \label{eq:t2amplrestricmp2}
\end{equation}

In Ref. \cite{Kristensen2011} the first guess of the $t2$ amplitude is 0 which
gives the residual
\begin{equation}
  R^{1,MP2}_{aibj}(t^1)=-g_{aibj}.
  \label{eq:MP2inital}
\end{equation}

In the first iteration the non-vanishing integrals $g_{aibj}$ for
$i\in\underline{P}$ and $j\in\underline{Q}$ occur only if
$a\in\{\overline{P\}}$ and $b\in\{\overline{Q}\}$. Following this line the
second iteration amplitude $t^{2,ab}_{ij}=R^{1,ab}_{ij}$ is nonvanishing only
when

\begin{equation}
  i\in\underline{P},a\in[\overline{P}],~j\in\underline{Q},b\in[\overline{Q}].
  \label{eq:2restt2mp2}
\end{equation}

The second iteration residual, $R^{2,MP2}_{aibj}$ is given by
\begin{equation}
  R^{2,MP2}_{aibj}=-g_{aibj}-\sum_{c\in[\overline{P}]}t^{2,cb}_{ij}F_{ac}-\sum_{c\in[\overline{Q}]}t^{2,ac}_{ij}F_{bc}+\sum_{k\in\underline{P}}t^{2,ab}_{kj}F_{ki}+\sum_{k\in\underline{Q}}t^{2,ab}_{ik}F_{kj},
  \label{eq:2itmp2res}
\end{equation}

and becomes nonnegligible when $a\in[\overline{P}]_2$, $b\in[\overline{Q}]_2$, for
the second and third term respectively and for $i\in[\underline{P}]$ and $j\in[\underline{Q}]$ in the fourth and fifth term respectively.
Here $[P]_2$ is the set of molecular orbitals including and local to $[P]$. 

Following the same procedure for the third iteration, the third residual
\begin{equation}
  R^{3,MP2}_{aibj}=-g_{aibj}-\sum_{c}t^{3,cb}_{ij}F_{ac}-\sum_{c}t^{3,ac}_{ij}F_{bc}+\sum_{k}t^{3,ab}_{kj}F_{ki}+\sum_{k}t^{3,ab}_{ik}F_{kj}
  \label{eq:thirdresid}
\end{equation}
becomes non-negligible when $a\in[\overline P]_3$, seen from the second term, $b\in[\overline Q]_3$, as seen from the third term, $i\in[\underline{P}]_2$ as seen from the fourth term and for $j\in [\underline{Q}]_2$ as seen from the fifth term in Eq. \eqref{eq:thirdresid}.
The following iteration will increase the fragmental space taking contributing 
to the amplitudes in the same manner.
The second iteration introduces a first order interaction (FOI) space containing
the amplitudes
\begin{equation}
  i\in[\underline{P}]-\underline{P},a\in[\overline{P}]_2-[\overline{P}],~j\in[\underline{Q}]-\underline Q,b\in[\overline{Q}]_2-[\overline{Q}].
  \label{eq:FOI}
\end{equation}

The second order interaction space (SOI) is similarily defined as
\begin{equation}
  i\in[\underline{P}]_2-[\underline{P}],a\in[\overline{P}]_3-[\overline{P}]_2,~j\in[\underline{Q}]_2-[\underline{Q}],b\in[\overline{Q}]_3-[\overline{Q}]_2,
  \label{eq:SOI}
\end{equation}

which is the difference in spaces between iteration four and three.
From Eq. \eqref{eq:2itmp2res} we notice that the terms referencing only the indices $a\in[\overline{P}]_2-[\overline{P}]$ and $b\in[\overline{Q}]_2-[\overline{Q}]$ are small because the residual has no integral contribution, whil the terms referencing $i\in[\underline{P}]-\underline{P}$ and $j\in[\underline{Q}]-\underline{Q}$ are on the same order as the first iteration. 
It follows then that the effect the SOI space has on the amplitude can in most cases be neglected. For the later iterations the new orbital spaces insignificant.






\bibliographystyle{h-physrev3}
\bibliography{DEC.bib}

\end{document}
